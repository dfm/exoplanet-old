\documentclass[12pt,preprint]{aastex}

% has to be before amssymb it seems
\usepackage{color,hyperref}
\definecolor{linkcolor}{rgb}{0,0,0.5}
\hypersetup{colorlinks=true,linkcolor=linkcolor,citecolor=linkcolor,
            filecolor=linkcolor,urlcolor=linkcolor}
\newcommand{\numberparagraphs}{}
\newcommand{\nonumberparagraphs}{}

\usepackage{url}
\usepackage{amssymb,amsmath}

\newcommand{\arxiv}[1]{\href{http://arxiv.org/abs/#1}{arXiv:#1}}
\newcommand{\project}[1]{{\sffamily #1}}
\newcommand{\foreign}[1]{\emph{#1}}
\newcommand{\etal}{\foreign{et\,al.}}
\newcommand{\etc}{\foreign{etc.}}
\newcommand{\unit}[1]{\mathrm{#1}}

\newcommand{\Fig}[1]{Figure~\ref{fig:#1}}
\newcommand{\fig}[1]{\Fig{#1}}
\newcommand{\figlabel}[1]{\label{fig:#1}}
\newcommand{\Tab}[1]{Table~\ref{tab:#1}}
\newcommand{\tab}[1]{\Tab{#1}}
\newcommand{\tablabel}[1]{\label{tab:#1}}
\newcommand{\Eq}[1]{Equation~(\ref{eq:#1})}
\newcommand{\eq}[1]{\Eq{#1}}
\newcommand{\eqlabel}[1]{\label{eq:#1}}
\newcommand{\Sect}[1]{Section~\ref{sect:#1}}
\newcommand{\sect}[1]{\Sect{#1}}
\newcommand{\App}[1]{Appendix~\ref{sect:#1}}
\newcommand{\app}[1]{\App{#1}}
\newcommand{\sectlabel}[1]{\label{sect:#1}}
\newcommand{\Algo}[1]{Algorithm~\ref{algo:#1}}
\newcommand{\algo}[1]{\Algo{#1}}
\newcommand{\algolabel}[1]{\label{algo:#1}}

% math symbols
\newcommand{\dd}{\mathrm{d}}
\newcommand{\like}{\mathscr{L}}
\newcommand{\bvec}[1]{\boldsymbol{#1}}
\newcommand{\paramvector}[1]{\bvec{#1}}
\newcommand{\normal}[2]{\mathcal{N} (#1, #2)}
\renewcommand{\vector}[1]{#1}
\renewcommand{\matrix}[1]{#1}
\newcommand{\pr}[1]{\ensuremath{p(#1)}}

% Parameters
\newcommand{\duration}{\ensuremath{\Delta t}}
\newcommand{\period}{\ensuremath{T}}
\newcommand{\radius}{\ensuremath{R}}
\newcommand{\orbitpars}{\ensuremath{\omega}}

\begin{document}

\title{Notes about transit modeling}

\newcommand{\nyu}{2}
\author{Daniel~Foreman-Mackey\altaffilmark{1,\nyu}}
\altaffiltext{1}{To whom correspondence should be addressed:
                        \url{danfm@nyu.edu}}
\altaffiltext{\nyu}{Center for Cosmology and Particle Physics,
                        Department of Physics, New York University,
                        4 Washington Place, New York, NY, 10003, USA}

\begin{abstract}

\end{abstract}

\keywords{}

In a single star, transits are observed with duration \duration\ and
period \period.  What can you infer about the radius \radius\ of
the secondary given these two observations?  In a large set of $N$
stars $n$, a fraction $f$ are found to show transits.  In each
transiting system there is transit duration $\duration_n$ and period
$\period_n$.  What can you infer about the distribution
\pr{\radius,\orbitpars} of secondary radii $\radius$ and orbital
parameters \orbitpars\ (a set of parameters, including semi-major
axis and eccentricity, for example, or energy and angular momentum)
for the whole population of primaries?

\begin{thebibliography}{}\raggedright

\bibitem[Batalha \etal(2012)]{Batalha:2012} Batalha, N.~M., Rowe,
    J.~F., Bryson, S.~T., \etal\ 2012, \arxiv{1202.5852}

\end{thebibliography}

\end{document}


